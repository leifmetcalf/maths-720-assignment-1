\documentclass[a4paper, 12pt]{article}
\usepackage[utf8]{inputenc}
\usepackage{mathtools}
\usepackage{amssymb}
\usepackage{enumitem}
\usepackage{parskip}
\usepackage{xfrac}
\usepackage{xcolor}

\newcommand{\tr}{^{\mathsf{T}}}
\newcommand{\N}{\mathbb{N}}
\newcommand{\R}{\mathbb{R}}
\newcommand{\Z}{\mathbb{Z}}
\newcommand{\Q}{\mathbb{Q}}
\newcommand{\half}{\sfrac{1\!}2}
\DeclarePairedDelimiter\abs{\lvert}{\rvert}
\DeclareMathOperator{\GL}{GL}
\DeclareMathOperator{\interior}{int}
\DeclareMathOperator{\closure}{cl}
\DeclareMathOperator{\sgn}{sgn}

\DeclareFontFamily{U}{mathb}{\hyphenchar\font45}
\DeclareFontShape{U}{mathb}{m}{n}{
      <5> <6> <7> <8> <9> <10> gen * mathb
      <10.95> mathb10 <12> <14.4> <17.28> <20.74> <24.88> mathb12
      }{}
\DeclareSymbolFont{mathb}{U}{mathb}{m}{n}
\DeclareFontSubstitution{U}{mathb}{m}{n}
\let\dddot\relax
\DeclareMathAccent{\dddot}{0}{mathb}{"3B}

\setlist[enumerate, 1]{leftmargin=0pt, label=\textbf{\arabic*.}}

\begin{document}

\begin{enumerate}

\item Let \(x,y\in G\). Then
\begin{align*}
&xyxy=1\\
\implies{}&xyxyyx=yx\\
\implies{}&xy=yx
\end{align*}
so \(G\) is Abelian.

\item We have \(q^n-1\) choices for a nonzero first column of \(\GL(n,F)\). For the second column we have \(q^n-q\) choices for a column not in the span of the first column. For the third column we have \(q^n-q^2\) choices for a column not in the span of the first two columns, and so on. Hence
\[\abs{\GL(n,F)}=\prod_{0\leq i<n}(q^n-q^i).\]

\item Let \(A\in Z(GL(n,F))\). Let \(E(i,j)\) denote the matrix with a one in position \((i,j)\) and zeroes everywhere else. Let \(1\leq i\leq n\) and \(1\leq j\leq n\). Then \(1+E(i,j)\in GL(n,F)\) and hence 
\begin{align*}
&A(1+E(i,j))=(1+E(i,j))A\\
\implies{}&AE(i,j)=E(i,j)A
\end{align*}
The left side is a matrix that is zero everywhere except the \(j\)th column which contains the \(i\)th column from \(A\), and the right side is a matrix that is zero everywhere except the \(i\)th row which contains the \(j\)th row from \(A\).

This implies the off-diagonal elements are zero, since to show \(A_{k,l}\) is zero with \(k\neq l\) we can pick \(i=l\) and \(j=k\). Then the nonzero elements of \(AE(i,j)\) and \(E(i,j)A\) coincide only at \((l,k)\neq(k,l)\).

It also implies the diagonal elements are equal to each other, since to show \(A_{k,k}=A_{l,l}\) we can pick \(i=k\) and \(j=l\). Then the nonzero elements of \(AE(i,j)\) and \(E(i,j)A\) coincide only at \((k,l)\) where \((AE(i,j))_{k,l}=A_{k,k}\) and \((E(i,j)A)_{k,l}=A_{l,l}\). Hence \(A\) is a scalar matrix.

\item Let \(A\coloneqq A_5\) and \(S\coloneqq S_5\) act by conjugation on \(\Omega\coloneqq A_5\). Let \(\alpha\in\Omega\). Suppose \(S_\alpha\) contains an odd permutation \(g\). Then \(\alpha^A=\alpha^S\) since if \(\alpha^h\in\alpha^S\) for some \(h\in S\) then either \(h\) is even and \(\alpha^h\in\alpha^A\) or \(h\) is odd and \(gh\) is even and \(\alpha^h=\alpha^{gh}\in\alpha^A\).

Conversely suppose \(S_\alpha\) does not contain an odd permutation. Then \(A_\alpha=S_\alpha\) and hence
\[\alpha^A\simeq\frac A{A_\alpha}=\frac A{S_\alpha}\]
and since
\begin{gather*}
\alpha^S\simeq\frac S{S_\alpha}\\
\abs{A}=\abs{S}
\end{gather*}
it follows that
\[\abs{\alpha^A}=\sfrac{1\!}2\abs{\alpha^S}.\]
Therefore if the centraliser of an element of a conjugacy class contains an odd permutation then that conjugacy class is equal to the corresponding class in \(S_5\); otherwise it is half its size.

There are five even conjugacy classes in \(S_5\):

The identity. Clearly this is a conjugacy class in \(A_5\) with one element.

Two disjoint 2-cycles. \((1,2)\) commutes with \((1,2)(3,4)\) so these are the same as in \(S_5\). There are \(\binom52\binom32/2!=15\) elements in this class.

3-cycles. \((4,5)\) commutes with \((1,2,3)\) so these are the same as in \(S_5\). There are \(\binom53\cdot2!=20\) elements in this class.

5-cycles. The only conjugations in \(S_5\) that preserve 5-cycles are 5-cycles, so this class splits into two. There are 24 5-cycles in \(S_5\) so there are two conjugacy classes of order 12 in \(A_5\) with representatives \((1,2,3,4,5)\) and \((1,2,3,5,4)\) respectively.

\item \begin{enumerate}

\item 

\item

\end{enumerate}

\item

\item \begin{enumerate}

\item

\item

\end{enumerate}

\item

\item

\end{enumerate}

\end{document}